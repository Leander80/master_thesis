Observing the Prompt Component of the Atmospheric Muon Flux Using IceCube

Atmospheric muons are created by the decays of secondary particles generated in cosmic ray interactions with the upper atmosphere. Based on the muons' parent particles, they can be categorized into conventional muons, originating from pions and kaons, and prompt muons, generated by the decays of more short-lived particles. While the conventional component dominates at lower energies, prompt muons become dominant at high energies, around 1 PeV and above.

Measuring these muons using the IceCube neutrino telescope is useful for studying hadronic interactions at a combination of center-of-mass energies and rapidities that are difficult to replicate in any current collider experiment. Due to the low overall flux at the energies where the prompt component dominates, no analysis to date has been able to significantly measure it.

The talk will cover the process of investigating the normalization of the prompt muon flux using a forward fit. This involves testing the method’s ability to identify the prompt component using simulations, as well as its intended subsequent application to actual IceCube data.